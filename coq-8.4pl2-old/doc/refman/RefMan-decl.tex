\newcommand{\DPL}{Mathematical Proof Language}

\chapter{The \DPL\label{DPL}\index{DPL}}

\section{Introduction}

\subsection{Foreword}

In this chapter, we describe an alternative language that may be used
to do proofs using the Coq proof assistant.  The language described
here uses the same objects (proof-terms) as Coq, but it differs in the
way proofs are described. This language was created by Pierre
Corbineau at the Radboud University of Nijmegen, The Netherlands.

The intent is to provide language where proofs are less formalism-{}
and implementation-{}sensitive, and in the process to ease a bit the
learning of computer-{}aided proof verification.

\subsection{What is a declarative proof ?{}}
In vanilla Coq, proofs are written in the imperative style: the user
issues commands that transform a so called proof state until it
reaches a state where the proof is completed. In the process, the user
mostly described the transitions of this system rather than the
intermediate states it goes through.

The purpose of a declarative proof language is to take the opposite
approach where intermediate states are always given by the user, but
the transitions of the system are automated as much as possible.

\subsection{Well-formedness and Completeness}

The \DPL{} introduces a notion of well-formed
proofs which are weaker than correct (and complete)
proofs. Well-formed proofs are actually proof script where only the
reasoning is incomplete. All the other aspects of the proof are
correct:
\begin{itemize}
\item All objects referred to exist where they are used
\item Conclusion steps actually prove something related to the
  conclusion of the theorem (the {\tt thesis}.
\item Hypothesis introduction steps are done when the goal is an
  implication with a corresponding assumption.
\item Sub-objects in the elimination steps for tuples are correct
  sub-objects of the tuple being decomposed.
\item Patterns in case analysis are type-correct, and induction is well guarded.
\end{itemize}

\subsection{Note for tactics users}

This section explain what differences the casual Coq user will
experience using the \DPL .
\begin{enumerate}
\item The focusing mechanism is constrained so that only one goal at
  a time is visible.
\item Giving a statement that Coq cannot prove does not produce an
  error, only a warning: this allows to go on with the proof and fill
  the gap later.
\item Tactics can still be used for justifications and after
{\texttt{escape}}.
\end{enumerate}

\subsection{Compatibility}

The \DPL{} is available for all Coq interfaces that use
text-based interaction, including:
\begin{itemize}
\item the command-{}line toplevel {\texttt{coqtop}}
\item the native GUI {\texttt{coqide}}
\item the Proof-{}General emacs mode
\item Cezary Kaliszyk'{}s Web interface
\item L.E. Mamane'{}s tmEgg TeXmacs plugin
\end{itemize}

However it is not supported by structured editors such as PCoq.



\section{Syntax}

Here is a complete formal description of the syntax for DPL commands.

\begin{figure}[htbp]
\begin{centerframe}
\begin{tabular}{lcl@{\qquad}r}
  instruction & ::= & {\tt proof} \\ 
  &  $|$ & {\tt assume } \nelist{statement}{\tt and}
  \zeroone{[{\tt and } \{{\tt we have}\}-clause]} \\
  &  $|$ & \{{\tt let},{\tt be}\}-clause \\
  &  $|$ & \{{\tt given}\}-clause \\
  &  $|$ & \{{\tt consider}\}-clause {\tt  from} term \\
  &  $|$ & ({\tt have} $|$ {\tt then} $|$ {\tt thus} $|$ {\tt hence}]) statement
  justification \\
  &  $|$ &  \zeroone{\tt thus} ($\sim${\tt =}|{\tt =}$\sim$) \zeroone{\ident{\tt :}}\term\relax justification \\  &  $|$ & {\tt suffices} (\{{\tt to have}\}-clause $|$
  \nelist{statement}{\tt and } \zeroone{{\tt and} \{{\tt to have}\}-clause})\\
  & & {\tt to show} statement justification \\ 
  &  $|$ & ({\tt claim} $|$ {\tt focus on}) statement \\
  &  $|$ & {\tt take} \term \\
  &  $|$ & {\tt define} \ident \sequence{var}{,}  {\tt  as} \term\\
  &  $|$ & {\tt reconsider} (\ident $|$ {\tt thesis}) {\tt  as} type\\
  &  $|$ & 
  {\tt per} ({\tt cases}$|${\tt induction}) {\tt on} \term \\
  &  $|$ & {\tt per cases of} type justification \\
  &  $|$ & {\tt suppose} \zeroone{\nelist{ident}{,} {\tt and}}~
  {\tt it is }pattern\\
  &          & \zeroone{{\tt such that} \nelist{statement} {\tt and} \zeroone{{\tt and} \{{\tt we have}\}-clause}} \\
  &  $|$ & {\tt end} 
  ({\tt proof} $|$ {\tt claim} $|$ {\tt focus} $|$ {\tt cases} $|$ {\tt induction}) \\
  & $|$ & {\tt escape} \\
  & $|$ & {\tt return} \medskip \\
  \{$\alpha,\beta$\}-clause & ::=& $\alpha$ \nelist{var}{,}~
  $\beta$ {\tt such that} \nelist{statement}{\tt and } \\
  & & \zeroone{{\tt and } \{$\alpha,\beta$\}-clause} \medskip\\
  statement   & ::= & \zeroone{\ident {\tt :}} type  \\
  & $|$ & {\tt thesis} \\
  & $|$ & {\tt thesis for} \ident \medskip \\
  var    & ::= & \ident \zeroone{{\tt :} type} \medskip \\
  justification & ::= & 
  \zeroone{{\tt by} ({\tt *} | \nelist{\term}{,})}
  ~\zeroone{{\tt using} tactic} \\
\end{tabular}
\end{centerframe}
\caption{Syntax of mathematical proof commands}
\end{figure}

The lexical conventions used here follows those of section \ref{lexical}.


Conventions:\begin{itemize}

 \item {\texttt{<{}tactic>{}}} stands for an Coq tactic.

 \end{itemize}

\subsection{Temporary names}

In proof commands where an optional name is asked for, omitting the
name will trigger the creation of a fresh temporary name (e.g. for a
hypothesis). Temporary names always start with an undescore '\_'
character (e.g. {\tt \_hyp0}). Temporary names have a lifespan of one
command: they get erased after the next command. They can however be safely in the step after their creation.

\section{Language description}

\subsection{Starting and Ending a mathematical proof}

 The standard way to use the \DPL is to first state a {\texttt{Lemma/Theorem/Definition}} and then use the {\texttt{proof}} command to switch the current subgoal to mathematical mode.  After the proof is completed, the {\texttt{end proof}} command will close the mathematical proof. If any subgoal remains to be proved, they will be displayed using the usual Coq display.

\begin{coq_example}
Theorem this_is_trivial: True.
proof.
  thus thesis.
end proof.
Qed.
\end{coq_example}

The {\texttt{proof}} command only applies to \emph{one subgoal}, thus
if several sub-goals are already present, the {\texttt{proof .. end
    proof}} sequence has to be used several times.

\begin{coq_eval}
Theorem T: (True /\ True) /\ True.
  split. split.
\end{coq_eval}
\begin{coq_example}
  Show.
  proof. (* first subgoal *)
    thus thesis.
  end proof.
  trivial. (* second subgoal *)
  proof. (* third subgoal *)
    thus thesis.
  end proof.
\end{coq_example}
\begin{coq_eval}
Abort.
\end{coq_eval}

As with all other block structures, the {\texttt{end proof}} command
assumes that your proof is complete. If not, executing it will be
equivalent to admitting that the statement is proved: A warning will
be issued and you will not be able to run the {\texttt{Qed}}
command. Instead, you can run {\texttt{Admitted}} if you wish to start
another theorem and come back
later.

\begin{coq_example}
Theorem this_is_not_so_trivial: False.
proof.
end proof. (* here a warning is issued *)
Qed. (* fails : the proof in incomplete *)
Admitted. (* Oops! *)
\end{coq_example}
\begin{coq_eval}
Reset this_is_not_so_trivial.   
\end{coq_eval}

\subsection{Switching modes}

When writing a mathematical proof, you may wish to use procedural
tactics at some point. One way to do so is to write a using-{}phrase
in a deduction step (see section~\ref{justifications}). The other way
is to use an {\texttt{escape...return}} block.

\begin{coq_eval}  
Theorem T: True.
proof.
\end{coq_eval}
\begin{coq_example}
 Show.
 escape.
 auto.
 return.
\end{coq_example}
\begin{coq_eval}
Abort.
\end{coq_eval}

The return statement expects all subgoals to be closed, otherwise a
warning is issued and the proof cannot be saved anymore.

It is possible to use the {\texttt{proof}} command inside an
{\texttt{escape...return}} block, thus nesting a mathematical proof
inside a procedural proof inside a mathematical proof ...

\subsection{Computation steps}

The {\tt reconsider ... as} command allows to change the type of a hypothesis or of {\tt thesis} to a convertible one.

\begin{coq_eval}
Theorem T: let a:=false in let b:= true in ( if a then True else False -> if b then True else False).
intros a b.
proof.
assume H:(if a then True else False).
\end{coq_eval}
\begin{coq_example}
 Show.
 reconsider H as False.
 reconsider thesis as True.
\end{coq_example}
\begin{coq_eval}
Abort.
\end{coq_eval}


\subsection{Deduction steps}

The most common instruction in a mathematical proof is the deduction step:
 it asserts a new statement (a formula/type of the \CIC) and tries to prove it using a user-provided indication : the justification. The asserted statement is then added as a hypothesis to the proof context.

\begin{coq_eval}
Theorem T: forall x, x=2 -> 2+x=4.
proof.
let x be such that H:(x=2).
\end{coq_eval} 
\begin{coq_example}
 Show.
 have H':(2+x=2+2) by H.
\end{coq_example}
\begin{coq_eval}
Abort.
\end{coq_eval}

It is very often the case that the justifications uses the last hypothesis introduced in the context, so the {\tt then} keyword can be used as a shortcut, e.g. if we want to do the same as the last example :
 
\begin{coq_eval}
Theorem T: forall x, x=2 -> 2+x=4.
proof.
let x be such that H:(x=2).
\end{coq_eval} 
\begin{coq_example}
 Show.
 then (2+x=2+2).
\end{coq_example}
\begin{coq_eval}
Abort.
\end{coq_eval}

In this example, you can also see the creation of a temporary name {\tt \_fact}.

\subsection{Iterated equalities}

A common proof pattern when doing a chain of deductions, is to do
multiple rewriting steps over the same term, thus proving the
corresponding equalities. The iterated equalities are a syntactic
support for this kind of reasoning:

\begin{coq_eval}
Theorem T: forall x, x=2 -> x + x = x * x.
proof.
let x be such that H:(x=2).
\end{coq_eval} 
\begin{coq_example}
 Show.
 have (4 = 4).
        ~= (2 * 2).
        ~= (x * x) by H.
        =~ (2 + 2).
        =~ H':(x + x) by H.
\end{coq_example}
\begin{coq_eval}
Abort.
\end{coq_eval}

Notice that here we use temporary names heavily.

\subsection{Subproofs}

When an intermediate step in a proof gets too complicated or involves a well contained set of intermediate deductions, it can be useful to insert its proof as a subproof of the current proof. this is done by using the {\tt claim ... end claim} pair of commands. 

\begin{coq_eval}
Theorem T: forall x, x + x = x * x -> x = 0 \/ x = 2.
proof.
let x be such that H:(x + x = x * x).
\end{coq_eval} 
\begin{coq_example}
Show.
claim H':((x - 2) * x = 0).
\end{coq_example}

A few steps later ...

\begin{coq_example}
thus thesis. 
end claim.
\end{coq_example}

Now the rest of the proof can happen.

\begin{coq_eval}
Abort.
\end{coq_eval}

\subsection{Conclusion steps}

The commands described above have a conclusion counterpart, where the
new hypothesis is used to refine the conclusion.

\begin{figure}[b]
 \centering
\begin{tabular}{c|c|c|c|c|}
        X       & \,simple\, & \,with previous step\, & 
               \,opens sub-proof\, & \,iterated equality\, \\
\hline
intermediate step & {\tt have} & {\tt then} & 
               {\tt claim} & {\tt $\sim$=/=$\sim$}\\ 
conclusion step & {\tt thus} & {\tt hence} & 
                {\tt focus on} & {\tt thus $\sim$=/=$\sim$}\\ 
\hline
\end{tabular}
\caption{Correspondence between basic forward steps and conclusion steps}
\end{figure}

Let us begin with simple examples :

\begin{coq_eval}
Theorem T: forall (A B:Prop), A -> B -> A /\ B.
intros A B HA HB.
proof.
\end{coq_eval} 
\begin{coq_example}
Show.
hence B.
\end{coq_example}
\begin{coq_eval}
Abort.
\end{coq_eval}

In the next example, we have to use {\tt thus} because {\tt HB} is no longer
the last hypothesis.

\begin{coq_eval}
Theorem T: forall (A B C:Prop), A -> B -> C -> A /\ B /\ C.
intros A B C HA HB HC.
proof.
\end{coq_eval} 
\begin{coq_example}
Show.
thus B by HB.
\end{coq_example}
\begin{coq_eval}
Abort.
\end{coq_eval}

The command fails the refinement process cannot find a place to fit
the object in a proof of the conclusion.


\begin{coq_eval}
Theorem T: forall (A B C:Prop), A -> B -> C -> A /\ B.
intros A B C HA HB HC.
proof.
\end{coq_eval} 
\begin{coq_example}
Show.
hence C. (* fails *)
\end{coq_example}
\begin{coq_eval}
Abort.
\end{coq_eval}

The refinement process may induce non
reversible choices, e.g. when proving a disjunction it may {\it
  choose} one side of the disjunction.

\begin{coq_eval}
Theorem T: forall (A B:Prop), B -> A \/ B.
intros A B HB.
proof.
\end{coq_eval} 
\begin{coq_example}
Show.
hence B.
\end{coq_example}
\begin{coq_eval}
Abort.
\end{coq_eval}

In this example you can see that the right branch was chosen since {\tt D} remains to be proved.

\begin{coq_eval}
Theorem T: forall (A B C D:Prop), C -> D -> (A /\ B) \/ (C /\ D).
intros A B C D HC HD.
proof.
\end{coq_eval} 
\begin{coq_example}
Show.
thus C by HC.
\end{coq_example}
\begin{coq_eval}
Abort.
\end{coq_eval}

Now for existential statements, we can use the {\tt take} command to
choose {\tt 2} as an explicit witness of existence.

\begin{coq_eval}
Theorem T: forall (P:nat -> Prop), P 2 -> exists x,P x.
intros P HP.
proof.
\end{coq_eval} 
\begin{coq_example}
Show.
take 2.
\end{coq_example}
\begin{coq_eval}
Abort.
\end{coq_eval}

It is also possible to prove the existence directly.

\begin{coq_eval}
Theorem T: forall (P:nat -> Prop), P 2 -> exists x,P x.
intros P HP.
proof.
\end{coq_eval} 
\begin{coq_example}
Show.
hence (P 2).
\end{coq_example}
\begin{coq_eval}
Abort.
\end{coq_eval}

Here a more involved example where the choice of {\tt P 2} propagates
the choice of {\tt 2} to another part of the formula.

\begin{coq_eval}
Theorem T: forall (P:nat -> Prop) (R:nat -> nat -> Prop), P 2 -> R 0 2 -> exists x, exists y, P y /\ R x y.
intros P R HP HR.
proof.
\end{coq_eval} 
\begin{coq_example}
Show.
thus (P 2) by HP.
\end{coq_example}
\begin{coq_eval}
Abort.
\end{coq_eval}

Now, an example with the {\tt suffices} command. {\tt suffices}
is a sort of dual for {\tt have}: it allows to replace the conclusion
(or part of it) by a sufficient condition. 

\begin{coq_eval}
Theorem T: forall (A B:Prop) (P:nat -> Prop), (forall x, P x -> B) -> A -> A /\ B.
intros A B P HP HA.
proof.
\end{coq_eval} 
\begin{coq_example}
Show.
suffices to have x such that HP':(P x) to show B by HP,HP'.
\end{coq_example}
\begin{coq_eval}
Abort.
\end{coq_eval}

Finally, an example where {\tt focus} is handy : local assumptions.  

\begin{coq_eval}
Theorem T: forall (A:Prop) (P:nat -> Prop), P 2 -> A -> A /\ (forall x, x = 2 -> P x).
intros A P HP HA.
proof.
\end{coq_eval} 
\begin{coq_example}
Show.
focus on (forall x, x = 2 -> P x).
let x be such that (x = 2).
hence thesis by HP.
end focus.
\end{coq_example}
\begin{coq_eval}
Abort.
\end{coq_eval}

\subsection{Declaring an Abbreviation}

In order to shorten long expressions, it is possible to use the {\tt
  define ... as ...} command to give a name to recurring expressions.

\begin{coq_eval}
Theorem T: forall x, x = 0 -> x + x = x * x .
proof.
let x be such that H:(x = 0).
\end{coq_eval} 
\begin{coq_example}
Show.
define sqr x as (x * x).
reconsider thesis as (x + x = sqr x).
\end{coq_example}
\begin{coq_eval}
Abort.
\end{coq_eval}

\subsection{Introduction steps}

When the {\tt thesis} consists of a hypothetical formula (implication
or universal quantification (e.g. \verb+A -> B+) , it is possible to
assume the hypothetical part {\tt A} and then prove {\tt B}. In the
\DPL{}, this comes in two syntactic flavors that are semantically
equivalent : {\tt let} and {\tt assume}. Their syntax is designed so that {\tt let} works better for universal quantifiers and {\tt assume} for implications.

\begin{coq_eval}
Theorem T: forall (P:nat -> Prop), forall x, P x -> P x.
proof.
let P:(nat -> Prop).
\end{coq_eval} 
\begin{coq_example}
Show.
let x:nat.
assume HP:(P x).
\end{coq_example}
\begin{coq_eval}
Abort.
\end{coq_eval}

In the {\tt let} variant, the type of the assumed object is optional
provided it can be deduced from the command. The objects introduced by
let can be followed by assumptions using {\tt such that}.

\begin{coq_eval}
Theorem T: forall (P:nat -> Prop), forall x, P x -> P x.
proof.
let P:(nat -> Prop).
\end{coq_eval} 
\begin{coq_example}
Show.
let x. (* fails because x's type is not clear *) 
let x be such that HP:(P x). (* here x's type is inferred from (P x) *)
\end{coq_example}
\begin{coq_eval}
Abort.
\end{coq_eval}

In the {\tt assume } variant, the type of the assumed object is mandatory but the name is optional :

\begin{coq_eval}
Theorem T: forall (P:nat -> Prop), forall x, P x -> P x -> P x.
proof.
let P:(nat -> Prop).
let x:nat.
\end{coq_eval} 
\begin{coq_example}
Show.
assume (P x). (* temporary name created *)
\end{coq_example}
\begin{coq_eval}
Abort.
\end{coq_eval}

After {\tt such that}, it is also the case :

\begin{coq_eval}
Theorem T: forall (P:nat -> Prop), forall x, P x -> P x.
proof.
let P:(nat -> Prop).
\end{coq_eval} 
\begin{coq_example}
Show.
let x be such that (P x). (* temporary name created *)
\end{coq_example}
\begin{coq_eval}
Abort.
\end{coq_eval}

\subsection{Tuple elimination steps}

In the \CIC, many objects dealt with in simple proofs are tuples :
pairs , records, existentially quantified formulas. These are so
common that the \DPL{} provides a mechanism to extract members of
those tuples, and also objects in tuples within tuples within
tuples...

\begin{coq_eval}
Theorem T: forall (P:nat -> Prop) (A:Prop), (exists x, (P x /\ A)) -> A.
proof.
let P:(nat -> Prop),A:Prop be such that H:(exists x, P x /\ A) .
\end{coq_eval} 
\begin{coq_example}
Show.
consider x such that HP:(P x) and HA:A from H.
\end{coq_example}
\begin{coq_eval}
Abort.
\end{coq_eval}

Here is an example with pairs:

\begin{coq_eval}
Theorem T: forall p:(nat * nat)%type, (fst p >= snd p) \/ (fst p < snd p).
proof.
let p:(nat * nat)%type.
\end{coq_eval} 
\begin{coq_example}
Show.
consider x:nat,y:nat from p.
reconsider thesis as (x >= y \/ x < y). 
\end{coq_example}
\begin{coq_eval}
Abort.
\end{coq_eval}

It is sometimes desirable to combine assumption and tuple
decomposition. This can be done using the {\tt given} command.

\begin{coq_eval}
Theorem T: forall P:(nat -> Prop), (forall n , P n -> P (n - 1)) -> 
(exists m, P m) -> P 0.
proof.
let P:(nat -> Prop) be such that HP:(forall n , P n -> P (n - 1)).
\end{coq_eval} 
\begin{coq_example}
Show.
given m such that Hm:(P m).  
\end{coq_example}
\begin{coq_eval}
Abort.
\end{coq_eval}
 
\subsection{Disjunctive reasoning}

In some proofs (most of them usually) one has to consider several
cases and prove that the {\tt thesis} holds in all the cases. This is
done by first specifying which object will be subject to case
distinction (usually a disjunction) using {\tt per cases}, and then specifying which case is being proved by using {\tt suppose}.


\begin{coq_eval}
Theorem T: forall (A B C:Prop), (A -> C) -> (B -> C) -> (A \/ B) -> C.
proof.
let A:Prop,B:Prop,C:Prop be such that HAC:(A -> C) and HBC:(B -> C).
assume HAB:(A \/ B).
\end{coq_eval} 
\begin{coq_example}
per cases on HAB.
suppose A.
  hence thesis by HAC.
suppose HB:B.
  thus thesis by HB,HBC.
end cases.
\end{coq_example}
\begin{coq_eval}
Abort.
\end{coq_eval}

The proof is well formed (but incomplete) even if you type {\tt end
 cases} or the next {\tt suppose} before the previous case is proved.

If the disjunction is derived from a more general principle, e.g. the
excluded middle axiom), it is desirable to just specify which instance
of it is being used :

\begin{coq_eval}
Section Coq.
\end{coq_eval}
\begin{coq_example}
Hypothesis EM : forall P:Prop, P \/ ~ P.
\end{coq_example} 
\begin{coq_eval}
Theorem T: forall (A C:Prop), (A -> C) -> (~A -> C) -> C.
proof.
let A:Prop,C:Prop be such that HAC:(A -> C) and HNAC:(~A -> C).
\end{coq_eval} 
\begin{coq_example}
per cases of (A \/ ~A) by EM.
suppose (~A).
  hence thesis by HNAC.
suppose A.
  hence thesis by HAC.
end cases.
\end{coq_example}
\begin{coq_eval}
Abort.
\end{coq_eval}

\subsection{Proofs per cases}

If the case analysis is to be made on a particular object, the script
is very similar: it starts with {\tt per cases on }\emph{object} instead.

\begin{coq_eval}
Theorem T: forall (A C:Prop), (A -> C) -> (~A -> C) -> C.
proof.
let A:Prop,C:Prop be such that HAC:(A -> C) and HNAC:(~A -> C).
\end{coq_eval} 
\begin{coq_example}
per cases on (EM A).
suppose (~A).
\end{coq_example}
\begin{coq_eval}
Abort.
End Coq.
\end{coq_eval}

If the object on which a case analysis occurs in the statement to be
proved, the command {\tt suppose it is }\emph{pattern} is better
suited than {\tt suppose}. \emph{pattern} may contain nested patterns
with {\tt as} clauses. A detailed description of patterns is to be
found in figure \ref{term-syntax-aux}. here is an example.

\begin{coq_eval}
Theorem T: forall (A B:Prop) (x:bool), (if x then A else B) -> A \/ B.
proof.
let A:Prop,B:Prop,x:bool.
\end{coq_eval} 
\begin{coq_example}
per cases on x.
suppose it is true.
  assume A.
  hence A.
suppose it is false.
  assume B.
  hence B.
end cases.
\end{coq_example}
\begin{coq_eval}
Abort.
\end{coq_eval}

\subsection{Proofs by induction} 

Proofs by induction are very similar to proofs per cases: they start
with {\tt per induction on }{\tt object} and proceed with {\tt suppose
  it is }\emph{pattern}{\tt and }\emph{induction hypothesis}. The
induction hypothesis can be given explicitly or identified by the
sub-object $m$ it refers to using {\tt thesis for }\emph{m}. 

\begin{coq_eval}
Theorem T: forall (n:nat), n + 0 = n.
proof.
let n:nat.
\end{coq_eval} 
\begin{coq_example}
per induction on n.
suppose it is 0.
  thus (0 + 0 = 0).
suppose it is (S m) and H:thesis for m.
  then (S (m + 0) = S m).
  thus =~ (S m + 0).
end induction.
\end{coq_example}
\begin{coq_eval}
Abort.
\end{coq_eval}

\subsection{Justifications}\label{justifications}


Intuitively, justifications are hints for the system to understand how
to prove the statements the user types in. In the case of this
language justifications are made of two components:

Justification objects : {\texttt{by}} followed by a comma-{}separated
list of objects that will be used by a selected tactic to prove the
statement. This defaults to the empty list (the statement should then
be tautological). The * wildcard provides the usual tactics behavior:
use all statements in local context. However, this wildcard should be
avoided since it reduces the robustness of the script.

Justification tactic : {\texttt{using}} followed by a Coq tactic that
is executed to prove the statement. The default is a solver for
(intuitionistic) first-{}order with equality.

\section{More details and Formal Semantics}

I suggest the users looking for more information have a look at the
paper \cite{corbineau08types}. They will find in that paper a formal
semantics of the proof state transition induces by mathematical
commands.
